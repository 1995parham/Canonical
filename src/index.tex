\section{Career development}

\textbf{How would you describe your level of experience as a professional software engineer?
What are your strengths as a software engineer?}

I started my carrier back in the university, there I wrote C code for embedded devices like Arduino and FGPA.
I wrote Kernel modules for communicating with FPGA Hardware interfcaces and I did implementation of FAT file system
for our monelight kernel project.
This increase my knowledge about the operating systems and networking. Then our lab got bigger and I had chance
to write an IoT Platform that communicate with embedded devices using LoRaWAN, Zigbee and NRF and has a ReSTFul
API for users. This push me towards backend development role. As a Backend Developer I mainly used Go and design
systems for handling heavy loads. During the time that I've worked as Backend Engineer I got chance to own my services
on Kubernetes which tought me lots of things about Infrastructure, Operators and Infrastructure as a Code.

Based on these experiences, I consider myself as a Senior software engineer and I think my main strengths is having
academic knowledge besides having working experience.

\textbf{What are your favourite software development environments, languages, and tools? Elaborate on your choices.}

I really enjoying \href{https://neovim.io/}{Neovim} and \href{https://www.vim.org/}{vim} and I use them for daily development.
Sometimes because of having different versions of software (specially when using Python) I switch to \href{VSCode}{https://code.visualstudio.com/}
and use Dev Containers (They are really great wehn you want to have same version of everything between your team members)

I am in love with Go and C/C++. I use Bash sepcially because of my dotfiles framework (Yes, I have framework for my dotfiles
and consistently doing a fresh install of my ArchLinux to test them). I had experience with
Python and Cython too. I use C for embedded device programming or when I need to implement something into
Kernel or Kernel modules. I use Go and Python for implementing CLI tools or services.

I really like to learn Rust but my knowledge now is limited to implement CLI tools.

\textbf{How extensive is your experience of C++ software engineering? Outline the applications that you have led in C++,
and your takeaways from that experience.}

I used C and C++ lots of time. I implement FAT file system in C which is my first big project using C. When I worked
at our university IoT team, I led a project for making a Link Layer protocol over nRF. This project is about implementing
an ARQ mechanisem which we used stop-and-wait, becuase we don't have many messages to deliver into devices and we want
to reduce the computation on the device side.
Then I worked with a team to implement a Dashboard for our home automation system using Qt Framework and C++. It was
touch dashboard we ran on our home panel which is powered by a Raspberry Pi 2.

The main issue for me was memory and threading management. For having a consistent view of our memory leak we did lots
of test with \href{Valgrind}{https://valgrind.org/}.

\textbf{Describe top Linux projects (incl. containers) you have worked on (purpose, market, etc.).
What are your contributions (design, implementation, testing, documentation, CI/CD, etc)?}

I worked on many cloud native services for a ride healing company. These services mostly handle the events (around
300k event per second) and are implemented in Go. My contribution is desig and then implementation mostly. Beside that
I designed our CI/CD pipelines for Gitlab and Github.

One my first production projects was writing a container migration tool using OpenVZ. In the SDN environment we wanted
to make our application resiliant and because of that we want to migrate them into another machine and then restart them.
Project back to 2013 and we choose OpenVZ to manage our containers. We provide a shared state repository (using Memcache)
and migrate application into different machines and they can easily track their state by reconnecting into the shared
repository.

One of the huge projects that I worked on, is Surge. Our company wants to change the ride price based on the demand.
I led the project from the begining, first we devide cities into districts and then use driver location events to find
out number of available drivers in each region. Then we use the get price event to figure out the number of passengers.
We have a lots of events on both of these channels, so we need to handle them properly. We devide service into two part,
one part consume these events and do the 15 minutes widowing (so it reduces the data volume a lot) and
push them into our NATS cluster and then second part reads these batch from our NATS cluster and calculate
the percentage that we need to change the price.

I have also contributed in the open source messaging system like NATS and EMQ. These contribution mostly happened
on their Operators and Helm charts.

\section{Experience}

\textbf{Please explain the differences between running an application natively, in a container, or VM.
How are containers useful in IoT deployments?}

When you run an application natively you need to have every library and dependencies of the app on your system.
Using VM or Container you can abstract them away and have a working application by just running it without
having any concerns about dependency conflict.
Containers are lighter than VM because they only abstract the file system layer, I mean using container
you are actually using the same kernel as the host.

VMs are use more resources when comapred to containers because they are emulating the whole system but they are
more secure because this seperation happen in lower layers of the system comared to containers in which they
are provided by operating systems.

When you want to change operating system settings or want to have different type of operating systems then you cannot
use containers.

One the issues that I have using IoT devices were upgrading their software. You have lots of devices out there
and you need to push a frameware upgrade to them. It is costly and in case of havnig issue you may need to get
your boards from customers and flash them which is not possible most of the time.

On our Home automation platform we resolved this issue using containers. We have Raspberry Pi board in each house
and it capable of running applications in docker container, so we just ask it to pull and run an image then
you can have the required application in your house. Even we can disable applications when you ask for or upgrade
them remotely just by asking to pull a new image without having any need to flash or etc.

This solution was a success for our team and the main drawback on that time actually was not having the same
option for our other boards (like Arduino Uno).

\textbf{Describe your experience with IoT and edge applications (hardware, operating system, networking, protocols, etc).}

I worked on IoT application in Home automation, agriculture and oil industry. I worked with networking protocols
like LoRaWAN, nRF, Zigbee, BLE, Cellular, 6LoWPAN etc.

I worked with operating systems like \href{https://archlinuxarm.org/}{Arch Linux ARM}, \href{https://www.raspbian.org/}{Raspbian},
\href{https://www.riot-os.org/}{Riot}, Ubuntu, Ubuntu Core

I worked on the Hardwars like Raspberry Pi, Nvidia Jetson Nano, Arduin (Uno, Doe, etc.), Zylinx Z-Turn, BeagleBone Black,
STM32F303, STM32F746.

Beside those, I had experience doing cross-completion using \href{https://github.com/crosstool-ng/crosstool-ng}{crosstool-ng}.

\textbf{Outline your thoughts on open source software. What is important to get right when you are working on open source projects?}

\textbf{Outline your thoughts on performance in software engineering. How do you ensure that your team's code is performant? Can you describe hard learning in this regard?}

I've worked in a company with millions request per day and during that time I learnt about the performance a lot.
Each commit may cause a memory leak or increase response time and we cannot check this using CI/CD for every commit.
Back then we setup a load testing pipeline for each release and we did a load test using \href{https://github.com/grafana/k6}{k6}.

\textbf{Outline your thoughts on documentation in software projects. What practices do you think software developers should follow in this regard? What are examples of the best open source documentation you have worked with?}
