\tableofcontents

\section{Career development}

\noindent
\textbf{How would you describe your level of experience as a professional software engineer?
What are your strengths as a software engineer?}

I started my carrier back in the university, there I wrote C code for embedded devices like Arduino and FPGA.
I wrote Kernel modules for communicating with FPGA Hardware interfcaces and I did implementation of FAT file system
for our Monolithic Kernel project.
This increased my knowledge about the operating systems and networking. Then our lab got bigger and I had a chance
to write an IoT Platform that communicates with embedded devices using LoRaWAN, Zigbee and nRF and has a RESTful
API for users. This pushed me towards backend development role. As a Backend Developer I mainly used Go and designed
systems for handling heavy loads. During the time that I've worked as Backend Engineer I got a chance to own my services
on \href{https://kubernetes.io/}{Kubernetes} which tought me lots of things about Infrastructure,
Operators and Infrastructure as a Code.

Based on 8+ years of experience in software engineering, I consider myself as a Senior software engineer and
I think my main strengths is having academic knowledge besides having working experience.
Also, as a service owner I was on-call and had many experience of doing fire fighting during the incidents so I have
tshooting as another strengths.

Recent years I got a chance to lead designing of the services that operate between more than 50 teams at our company and
during that time I've always used the latest version of Go and setup NATS in our company as a messaging system
for the first time and then doing workshops for our people to start using it. Then company needed to have machine
learning pipeline and I set it up using Airflow, Spark and Kafka. Because of these, I can consider adaptibility and
continues learning as my other strengths.

At the end, I always try to own products and learn more about their domain, so I can have more opinions about them.
For example when I worked at oil industry I did my best to learn everything about pumpjacks and oil wells because
writing a good software always requires domain knowledge.

\noindent
\textbf{What are your favourite software development environments, languages, and tools? Elaborate on your choices.}

I really enjoy \href{https://neovim.io/}{Neovim} and \href{https://www.vim.org/}{vim} and I use them for daily development
because they are easy to use and heavily customizable and I love to spending time to make my editor
(My configuration are available on GitHub, but I cannot reference them here because I want to this document be anonymous).
Sometimes because of having different versions of software (specially when using Python) I switch to \href{https://code.visualstudio.com/}{VSCode}
and use Dev Containers (They are really great when you want to have same version of everything between your team members)

I am in love with Go and C/C++. I use Bash sepcially because of my dotfiles framework (Yes, I have framework for my dotfiles
and consistently doing a fresh installation of my ArchLinux to test them and make sure they work as expected). I have experience with
Python and Cython too. I use C for embedded device programming or when I need to implement something into
Kernel or Kernel modules. I use Go and Python for implementing CLI tools or services.
Based on situations I had to write code in PHP, Scala, Java, NodeJS and even Perl.

I love C/C++ because they are low-level and using them you have control over everything. Using C/C++ you need to know
about Operating system, difference between Stack and Heap and I really enjoy using my knowledge when writing code.
The first time I saw Go, I thought of it as C but for writing high level services and started writing HTTP servers,
CLI tools, etc in Go. I used Go since then for use cases. I like Python too because
you can customize your object as much as you can (you can override operators like what you can do in C++).

Daily, I use \href{https://www.docker.com/}{Docker} and \href{https://podman.io/}{Podman} as container management tools,
Git as version control system, GitHub and GitLab and their CI/CD platforms, ArchLinux as main operating system, Ubuntu
on servers, KVM and libvirtd to have virtual machines on my system, MiniKube to launch kubernetes locally,
screen for connecting over serial interfaces, \href{https://systemd.io/}{Systemd} to manage my period jobs and services
on Linux (I use \href{https://www.launchd.info/}{launchd} in macOS when I am fored to use macOS),
\href{https://www.ansible.com/}{Ansible} and \href{https://www.terraform.io/}{Terraform} for managing systems using code,
\href{https://github.com/tmux}{tmux} as terminal multiplexer,
\href{https://github.com/alacritty}{Alacritty} as terminal emulator,
\href{https://swaywm.org/}{Sway} as windowing manager and \href{https://wayland.freedesktop.org/}{Wayland} as display server,
many CLI tools like \href{https://github.com/stern/stern}{stern},
\href{https://kubernetes.io/docs/reference/kubectl/}{kubectl}, \href{https://krew.sigs.k8s.io/}{krew},
\href{https://helm.sh/}{helm}, etc.

I prefer to run applications using containers instead of installing them because it removes the burden of installing
dependencies. I always prefer CLI tools over GUI to me and always prefer the truly open source projects.

As ArchLinux user, I maintain packages on Arch User Repository too for the open source tools that doesn't exist
in the official repositories.

\noindent
\textbf{How extensive is your experience of C++ software engineering? Outline the applications that you have led in C++,
and your takeaways from that experience.}

I used C and C++ lots of time.
I participated in \href{http://eudyptula-challenge.org/}{Eudyptula Challenge} and started my Kernel journey from there.
I implemented FAT file system in C which was my first big project using C. When I worked
at our university IoT team, I led a project for making a Link Layer protocol over nRF physical layer to make it stable.
This project is about implementing an ARQ mechanism which we chose stop-and-wait approach, becuase we didn't have many messages
to deliver into devices and we wanted to reduce the computation on the device side.
Then I worked with a team to implement a Dashboard for our home automation system
using \href{https://www.qt.io/product/framework}{Qt Framework} and C++. It was
a touch dashboard we ran on our home panel which is powered by a Raspberry Pi 2.
Then moved for another project for monitoring farms using LoRaWAN, we implemented the firmware code using C++ for reading
sensors and publishing data using LoRaWAN. On the server side we used \href{https://www.chirpstack.io/}{Chirpstack}
to gather data and published them into our IoT Platform which is writen by me again but in Go
(At the time of that project, Chirpstack was named loraserver.io).

On my recent role, we did some changes including adding cache, change request/response structure, etc
on our private forks of the \href{https://github.com/valhalla/valhalla}{Valhalla} and
\href{https://github.com/Project-OSRM/osrm-backend}{OSRM} routing engines which both are implemented in C++.

The main issue for me using C/C++ was memory and threading management.
For having a consistent view of our memory leaks we did lots of tests with \href{https://valgrind.org/}{Valgrind}.
Other things that I encountered with during my projects were different OS compatibility challenges,
Parsing protocols binary headers and endian issues, Cross compilation issues between Intel and ARM processors,
Allocating memory on systems without having any operating systems installed (like Arduino devices), etc.

\noindent
\textbf{Describe top Linux projects (incl. containers) you have worked on (purpose, market, etc.).
What are your contributions (design, implementation, testing, documentation, CI/CD, etc)?}

I worked on many cloud native services for a ride healing company. These services mostly handle the events (around
300k event per second) and are implemented in Go. My contributions were desgining and then implementation mostly.
Beside that I designed our CI/CD pipelines in GitLab and GitHub.

One my first production projects was writing a container migration tool using \href{https://openvz.org/}{OpenVZ}.
In the SDN environment we wanted to make our application resiliant and because of that we wanted to migrate them into
another machine and then restart them in case any issue with their current bare metal.
Project goes back to 2013 and we chose OpenVZ to manage our containers. We provided a shared state repository (using \href{https://memcached.org/}{Memcached})
and migrated application into different machines and they cloud easily tracked their state by reconnecting into the shared
repository.

One of the huge projects that I worked on, is Surge. Our company wants to change the ride price based on the demand.
I led the project from the begining, first we devided cities into districts and then used driver location events to find
out number of available drivers in each region. Then we used the get price events to figure out the number of passengers.
We had lots of events on both of these channels, so we needed to handle them properly. We devided the service into two parts,
one part consumes these events and does the 15 minutes windowing (so it reduces the data volume to a great extend) and
pushes them into our NATS cluster and then second part reads these batches from our NATS cluster and calculates
the percentage that we need to change the price.
Testing the surge project is done by the QA team but me and our team helped them during the testing by providing
the required environment on Kubernetes and the conditions they wanted.
continues documentation for this feature was done from the begining on Confluence by the product team and we updated it
based on the implementation at the end of each phase of the project.
The code and design also were documented in the code base and GitLab wiki.

I also designed the machine learning pipeline for our company. This pipeline gathers data from our NATS cluster,
moves the data into Kafka (which is set up by us using \href{https://strimzi.io/}{Strimzi} Operator)
and then we process the data using Kafka Stream and feeds them into Spark applications and Cassandra for the machine learning
team. This project required a very close coorporation between us and the machine learning team to test and make
sure we have all required data. We needed to make sure we don't lose any data here and for that we used blackbox monitoring
by writing a consumer that checks the number of recieved events with database.

I have also contributed in the open source messaging system like
\href{https://github.com/nats-io/}{NATS} and \href{https://github.com/emqx}{EMQX}.
These contribution mostly happened on their
\href{https://kubernetes.io/docs/concepts/extend-kubernetes/operator/}{Operators}
and \href{https://helm.sh/}{Helm} charts.

\section{Experience}

\noindent
\textbf{Please explain the differences between running an application natively, in a container, or VM.
How are containers useful in IoT deployments?}

When you run an application natively you need to have every libraries and dependencies of the app installed on your system.
Using VM or Container you can abstract them away and have a working application by just running it without
having any concerns about dependency conflicts.
Containers are lighter than VM because they only abstract the file system, networking, timezone and other namespaced
features in Linux kernel (cotainers are based on cgroup, namespace and union filesystem of Linux Kernel).
I mean using container means you are actually using the same kernel as the host running that container.

VMs use more resources when compared to containers because they are emulating the whole system,
but they are more secure because this separation happen in lower layers of the system compared
to containers in which they are provided by operating systems.

When you want to change operating system settings or want to have different type of operating systems then you cannot
use containers.

One of the issues that I had using IoT devices were upgrading their software.
You have lots of devices out there, and you need to push a firmware upgrade to them.
It is costly and in case of having any problem you may need to get your boards from customers and flash them
which is not possible most of the time.
On our Home automation platform we resolved this issue using containers. We had a Raspberry Pi board in each house
and it was capable of running applications in docker containers, so we just asked it to pull and run an image then
you can have the required application in your house. Even we could disable applications when you asked for or upgrade
them remotely just by asking to pull a new image without having any need to flash or etc.
This solution was a success for our team and the main drawback on that time actually was not having the same
option for our other boards (like Arduino Uno).

\noindent
\textbf{Describe your experience with IoT and edge applications (hardware, operating system, networking, protocols, etc).}

I worked on IoT application in Home automation, agriculture and oil industry. I worked with networking protocols
like LoRaWAN, nRF, Zigbee, BLE, Cellular, 6LoWPAN etc.
On the Application layer I worked with RESTful APIs, CoAP and MQTT.
I used MQTT for gathering information from devices
(sometime we need to have gateway for converting their protocol into MQTT). For MQTT I have experience with
\href{https://github.com/emqx/emqx}{EMQX} (both on IoT and non-IoT industries), we set it up to hanndle more than 400k connections
at the same time on Kubernetes and recently we switched to their latest version (v5) which had great improvement on our performance.

I worked with operating systems like \href{https://archlinuxarm.org/}{Arch Linux ARM}, \href{https://www.raspbian.org/}{Raspbian},
\href{https://www.riot-os.org/}{Riot}, Ubuntu, Ubuntu Core, \href{https://www.freertos.org/index.html}{FreeRTOS},
\href{https://os.mbed.com/}{ARM Mbed} on IoT devices.

I worked on the Hardwars like Raspberry Pi, Nvidia Jetson Nano, Arduino (Uno, Doe, etc.), Zylinx Z-Turn, BeagleBone Black,
STM32F303, STM32F746.

Beside those, I had experience doing cross-completion using \href{https://github.com/crosstool-ng/crosstool-ng}{crosstool-ng}.

\noindent
\textbf{Outline your thoughts on open source software.
What is important to get right when you are working on open source projects?}

Opne source softwares are the place that people are really free and can
work together without paying attention to their nationalities, reglion, etc. and because of
that open source project grow really fast.

I created an open source organization for each company I worked for and leading
the OSS division of the company to release the tools that are useful in GitHub. I think by helping open source world
you will get feedback and see is if your software really needed or not.

The main thing for me in an open source project is the documentation, people need to know how to use your software,
how they can extend it, etc. On the next level you need to always maitain the project or archive it because
when people use it they need the bugs be to resolved, issues be to addressed, etc.

Beside our duties, people may need features that are not in our roadmap or scope and we need to address these very
clearly and say for example we are not resolving it for now but may in future or we think this feature is not in the
scope of this project.

\noindent
\textbf{Outline your thoughts on performance in software engineering. How do you ensure that your team's code is performant?
Can you describe hard learning in this regard?}

I've worked in a company with millions request per day and during that time I learnt about the performance a lot.
Each commit may cause a memory leak or increase response time and we cannot check this using CI/CD for every commit.
Back then we setup a load testing pipeline for each release and we did a load test using \href{https://github.com/grafana/k6}{k6}.

After getting the approval and doing the release we need to monitor and make sure we are in a good shape.
For monitoring we used \href{https://prometheus.io/}{Prometheus} and \href{https://grafana.com/}{Grafana} to monitor
metrics (that we exporeted in the application). We used histogram for response time (and the monitor its p50, p90 and p99),
we had counters for number of requests that we are processing in the moment, etc.

Using these metrics we can figure out issues but we need to have more detail view of the problem so we can resolve it.
For getting more insight we used \href{https://pyroscope.io/}{Pyroscope} and tracing. By tracing we can figure out where are the bottlenechs
and the do our best to resolve them. Pyroscope actually useful when we need way more detail about the problem
and we only ran it in our staging environments.

There was time that our team consider these tools as a fancy tools and then on production we got into the trobble,
that our queries takes more than 30 seconds and AWS forcibly close our connections then we ran into the problem of
find which query takes longer than other and this situation taught us these tools are not facny we alway needing them
even for a small B2B project.

\noindent
\textbf{Outline your thoughts on documentation in software projects. What practices do you think software developers
should follow in this regard? What are examples of the best open source documentation you have worked with?}

Documentation should happen in code, so we can make sure they are updated. The outdated documentation is worst than
not having the documentation because outdated one make people to do the things in a wrong way instead of at least
asking from someone else.

I personally prefer to use comments in the code and make sure they are updated in the review session. Other kind of
documentation can happen in Wiki so we can make sure they are updated one in a while when we want to do a release.

Non-technical people may prefer Confluence (or other documentation tools) so I think we can stick to them only for
non-technical documentation about the features and the way want them to work.

In the IoT world, if possible it would be good to use Protocols that has schema built-in their nature like ProtoBuf or
Apache Avro (which is used by \href{https://github.com/kaaproject/kaa}{Kaa} as far as I remember). Using this protocols
remove the need of having another document for schema and people can safely using those in their code.

\section{Education}

\noindent
\textbf{How did you rank in your final year of high school in mathematics? Were you a top student? On what basis would you say that?}

I was a top student (among 10\%) for the matmathmatics and computer programming.

\noindent
\textbf{How did you rank in your final year of high school, in your home language? Were you a top student? On what basis would you say that?}

Our home language is Persian and no I was not a top student.

\noindent
\textbf{Please state your high school graduation results or university entrance results, and explain the grading system used. For example, in the US, you might give your SAT or ACT scores.
In Germany, you might give your scores out of a grading system of 1-5, with 1 being the best.}

In Iran we have an enternance exam in which I ranked 646 among all of the students at that year (almost 100,000 students)
and it resulted me to enter Amirkabir University which is the third university in our country.

\noindent
\textbf{Can you make a case that you are in the top 5\% in your academic year, or top 1\%, or even higher?
If so please outline that case. Make reference where possible to standardised testing results at regional or national level, or university entrance results. Please explain any specific grading system used.}

During my M.Sc. and B.Sc. I was top student in my class and got graduated with 18.94 out of 20 for Bachelor and 18.77 out of 20 for Master.
(references exist upon request)

\noindent
\textbf{What sort of high school student were you? Outside of class, what were your interests and hobbies? What would your high school peers remember you for?}

I am bookworm person since high school and I spent all of my free time in the high school library.
Beside reading books, I really love networking and always friends remember me for making their network to play Counter Strike.

\noindent
\textbf{Which university and degree did you choose? What other universities did you consider, and why did you select that one?}

During high school I fell in love with computers (specially networking) and because of that I chose to pursue
it in the university too. Among the university I choosed Amirkabir University of Tech. because of my rank in the enterance exam,
for going into two other universities (that have better rank compare to Amirkabir) I need a better rank.

\noindent
\textbf{Overall, what was your degree result and how did that reflect on your ability? Please help us understand the grading system for your results.}

I was the first in my class during Bachelor and Master, and I think it happened because I really love computers,
operating systems and networking. I think the knowledge I grasped in the university is helping me a lot, I always
try to understand the problems first and make think logical before approaching them because computers are logical
and don't do anything random, and this approach always helping me to figure out the root cause.

\noindent
\textbf{During all of your education years, from high school to university, can you describe any achievements that were truly exceptional?}

During Bachelor, I participated in National Olympiad of Computer Networking and I ranked 4 in the first round and ranked
10 during the second round.
During Highschool, I participated in national Olympiad in Informatics and I only can passed two round of the exams, so
I didn't choice to participate in International Olympiad in Informatics from our country.

\noindent
\textbf{What leadership roles did you take on during your education?
Did you conceive of, and drive to completion, any initiatives outside of your required classwork?}

During my education, I led technical parts of the events that we were held in our college of Computer Engineering including
Online and Onsite Programming competition, Linux and Open source festivals, etc.

Beside those events, I led an open source project in which we used \href{https://freeradius.org/}{FreeRadius}
to control the internet usage of students, professors, etc.

I really love teaching and starting from Bachelor I was teaching assistent for at least for one course.
After finishing my Master, uinversity invited me to teach some technical courses like C Programming and
Internet Engineering. Since then I try to teach at least one of these course each semester.

\section{Context}

\noindent
\textbf{Outline your thoughts on the mission of Canonical. What is it about the company's purpose and goals which is most appealing to you?
What is it that you think is risky or unappealing?}

\noindent
\textbf{What would you like to achieve in career and skills development?}

I really love to continue my career in IoT and embedded devices. Also, I pasioned about writing IoT Platforms
(Because I wrote one years ago and contributed in many including \href{https://github.com/mainflux}{Mainflux},
\href{https://github.com/kaaproject}{Kaa}, \href{https://github.com/thingsboard}{ThingsBoard}, etc.)
and this position is really intersting to me because of being IoT and platform related.
I enjoy resolving challenges and I always try to challenge myself, and I think working at Canonical will be my next challenge.

Another major thing for me is having effect on the open source world and I think by working at Canonical I can
have more impact on the open source world.
